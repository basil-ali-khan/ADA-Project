\documentclass[a4paper,12pt]{article}
\usepackage[a4paper,margin=1in]{geometry}
\usepackage{hyperref}
\usepackage{graphicx}

\title{Paper Selection Proposal}
\author{Basil Ali Khan \and Ahsan Siddiqui}
\date{\today}

\begin{document}

\maketitle

\section*{Paper Details}
\begin{itemize}
    \item \textbf{Title:} \textit{GAGAN: Enhancing Image Generation Through Hybrid Optimization of Genetic Algorithms and Deep Convolutional Generative Adversarial Networks}
    \item \textbf{Authors:} Despoina Konstantopoulou, Paraskevi Zacharia, Michail Papoutsidakis, Helen C. Leligou, Charalampos Patrikakis
    \item \textbf{Conference/Journal:} \textit{Algorithms} (ISSN 1999-4893)
    \item \textbf{Year:} 2024
    \item \textbf{DOI/Link:} \href{https://doi.org/10.3390/a17120584}{https://doi.org/10.3390/a17120584}
\end{itemize}

\section*{Summary}
This paper presents \textbf{GAGAN}, a hybrid approach that enhances \textbf{Generative Adversarial Networks (GANs)} by integrating \textbf{Genetic Algorithms (GAs)} to optimize the discriminator's weights. Traditional \textbf{Deep Convolutional GANs (DCGANs)} often suffer from training instability and mode collapse. By incorporating evolutionary techniques like crossover and mutation, GAGAN improves convergence stability and image quality. The model was tested on the \textbf{CelebA dataset}, generating high-quality \textbf{128 × 128 images}. The results showed lower generator loss and better image fidelity compared to standard DCGANs.
\begin{figure}
    \centering
    \includegraphics[width=0.5\linewidth]{image.png}
    \caption{GAGAN Model Flowchart}
    \label{fig:enter-label}
\end{figure}

\section*{Justification}
This paper is highly relevant to our project because:
\begin{enumerate}
    \item It explores a novel \textbf{hybrid learning} approach that combines \textbf{deep learning} and \textbf{evolutionary algorithms}.
    \item It tackles key challenges in \textbf{GAN training} (mode collapse, instability), making it crucial for advancing generative models.
    \item The integration of \textbf{Genetic Algorithms} in \textbf{deep learning} is an emerging research area with real-world applications in \textbf{image generation} and \textbf{AI creativity}.
\end{enumerate}

\section*{Implementation Feasibility}
\begin{itemize}
    \item \textbf{Code \& Resources:} The paper does not mention a public implementation, but we plan to implement the genetic algorithms ourselves and using built-in libraries for the neural network training part as discussed in the meeting with our instructor.
    \item \textbf{Datasets:} The CelebA dataset is publicly available and can be used for training. However, we might preferably choose to use another publicly available dataset
    \item \textbf{Computational Requirements:} Training GANs requires \textbf{high-end GPUs}, hence we will be using Google Colab for this project.
\end{itemize}

\section*{Team Responsibilities}
We plan to divide work equally at every stage. Since Ahsan has previously worked in this domain, he will be taking the lead in the implementation stage. 
\begin{itemize}
    \item \textbf{Reading \& Understanding:} Basil Ali Khan, Ahsan Siddiqui
    \item \textbf{Coding \& Implementation:} Ahsan Siddiqui, Basil Ali Khan
    \item \textbf{Writing \& Report Preparation:} Ahsan Siddiqui, Basil Ali Khan
\end{itemize}

\section*{GitHub Repository}
The implementation and project code will be maintained at:  
\href{https://github.com/basil-ali-khan/ADA-Project}{https://github.com/basil-ali-khan/ADA-Project}


\end{document}
